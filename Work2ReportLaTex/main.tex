%!TeX program = xelatex
\documentclass[12pt,hyperref,a4paper,UTF8]{ctexart}
\usepackage{HDUReport}
\usepackage{listings}
\usepackage{xcolor}

\usepackage{setspace}
\setstretch{1.5} % 设置全局行距为1.5倍

\usepackage{enumitem} % 载入enumitem包以便自定义列表环境
\setlist[itemize]{itemsep=0pt, parsep=0pt} % 设置itemize环境的项目间距和段落间距

\setmainfont{Times New Roman} % 英文正文为Times New Roman

%封面页设置
{   
    %标题
    \title{ 
        \vspace{1cm}
        \heiti \Huge \textbf{Linux系统及应用作业报告} \par
        \vspace{1cm} 
        \heiti \Large {\underline{作业2:中国手机操作系统发展综述}   } 
        \vspace{3cm}
    
    }

    \author{
        \vspace{0.5cm}
        \kaishu\Large 学院\ \dlmu[9cm]{卓越学院} \\ %学院
        \vspace{0.5cm}
        \kaishu\Large 学号\ \dlmu[9cm]{23040447} \\ %班级
        \vspace{0.5cm}
        \kaishu\Large 姓名\ \dlmu[9cm]{陈文轩} \qquad  \\ %学号
        \vspace{0.5cm}
        \kaishu\Large 专业\ \dlmu[9cm]{智能硬件与系统(电子信息工程)} \qquad \\ %姓名 
    }
        
    \date{\today} % 默认为今天的日期,可以注释掉不显示日期
}
%%------------------------document环境开始------------------------%%
\begin{document}

%%-----------------------封面--------------------%%
\cover
\thispagestyle{empty} % 首页不显示页码
%%------------------摘要-------------%%
%\newpage
%\begin{abstract}




%\end{abstract}

%\thispagestyle{empty} % 首页不显示页码

%%--------------------------目录页------------------------%%
% \newpage
% \tableofcontents
% \thispagestyle{empty} % 目录不显示页码

%%------------------------正文页从这里开始-------------------%
\newpage
\setcounter{page}{1} % 让页码从正文开始编号

%%可选择这里也放一个标题
%\begin{center}
%    \title{ \Huge \textbf{{标题}}}
%\end{center}

\section{发展历程与技术演进}
中国手机操作系统发展可分为三个阶段:

\subsection{技术引进期(2008-2012)}
\begin{itemize}
    \item 中国移动基于Android 1.5开发OMS系统(2009),预装于联想OPhone,首次实现TD-SCDMA终端适配
    \item 阿里YunOS采用自主云应用框架(2011),但因兼容性问题导致与谷歌发生冲突\cite{yunos2012}
\end{itemize}

\subsection{生态培育期(2013-2018)}
\begin{itemize}
    \item 华为推出EMUI 1.0(2013),建立首个国产深度定制系统,全球用户突破5亿
    \item 阿里YunOS 5.0搭载于魅族MX4(2014),市占率曾达7%后因专利纠纷衰退
\end{itemize}

\subsection{自主创新期(2019至今)}
\begin{itemize}
    \item 鸿蒙OS 1.0发布(2019),首次实现分布式架构,代码自主率超80%
    \item OpenHarmony 3.0 LTS版本发布(2022),形成完整物联网能力矩阵
\end{itemize}

\begin{figure}[htbp]
    \centering
    \includegraphics[width=0.8\textwidth]{timeline.pdf}
    \caption{中国手机操作系统里程碑事件}
    \label{fig:timeline}
\end{figure}

\section{产业协同模式}
\subsection{手机厂商合作机制}
\begin{table}[htbp]
    \centering
    \caption{主流国产系统合作厂商(2024)}
    \begin{tabular}{lll}
    \toprule
    操作系统 & 核心合作伙伴 & 装机量(万) \\
    \midrule
    鸿蒙OS & 华为、荣耀、格力 & 42,000 \\
    OpenHarmony & 美的、九阳、传音 & 8,500 \\
    澎湃OS & 小米 & 12,000 \\
    \bottomrule
    \end{tabular}
\end{table}

\subsection{运营商深度参与}
\begin{itemize}
    \item 中国移动"泛终端计划":2023年采购2000万台鸿蒙设备
    \item 中国广电5G NR广播与鸿蒙深度集成,时延降低40%
\end{itemize}

\section{技术架构对比}
\begin{table}[htbp]
    \centering
    \caption{国内外主流系统技术对比}
    \begin{tabular}{llll}
    \toprule
     & 鸿蒙OS 4.0 & Android 14 & iOS 17 \\
    \midrule
    内核类型 & 微内核 & 宏内核 & XNU混合内核 \\
    分布式时延 & <20ms & 不支持 & 不支持 \\
    代码自主率 & 82\% & 0\% & 100\% \\
    \bottomrule
    \end{tabular}
\end{table}

\section{关键挑战}
\subsection{技术瓶颈}
\begin{itemize}
    \item 微内核性能损失:鸿蒙IPC时延比Android高15-30%
    \item 开发工具链不完善:DevEco Studio功能仅为Android Studio的65%\cite{huawei2023}
\end{itemize}

\subsection{生态困境}
\begin{figure}[htbp]
    \centering
    \includegraphics[width=0.6\textwidth]{ecosystem.pdf}
    \caption{全球移动应用生态覆盖率(2024Q1)}
    \label{fig:eco}
\end{figure}

\section{发展路径建议}
\begin{enumerate}
    \item \textbf{差异化竞争}:聚焦物联网场景(车载OS时延已优化至5ms)
    \item \textbf{标准建设}:主导IEEE 2050-202X分布式系统国际标准
    \item \textbf{人才培养}:教育部"智能基座"计划已覆盖72所高校
\end{enumerate}






\end{document}