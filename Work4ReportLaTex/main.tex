%!TeX program = xelatex
\documentclass[12pt,hyperref,a4paper,UTF8]{ctexart}
\usepackage{HDUReport}
\usepackage{listings}
\usepackage{xcolor}

\usepackage{setspace}
\setstretch{1.5} % 设置全局行距为1.5倍

\usepackage{enumitem} % 载入enumitem包以便自定义列表环境
\setlist[itemize]{itemsep=0pt, parsep=0pt} % 设置itemize环境的项目间距和段落间距

\setmainfont{Times New Roman} % 英文正文为Times New Roman

%封面页设置
{   
    %标题
    \title{ 
        \vspace{1cm}
        \heiti \Huge \textbf{Linux系统及应用作业报告} \par
        \vspace{1cm} 
        \heiti \Large {\underline{作业4:proc虚拟目录的作用}   } 
        \vspace{3cm}
    
    }

    \author{
        \vspace{0.5cm}
        \kaishu\Large 学院\ \dlmu[9cm]{卓越学院} \\ %学院
        \vspace{0.5cm}
        \kaishu\Large 学号\ \dlmu[9cm]{23040447} \\ %班级
        \vspace{0.5cm}
        \kaishu\Large 姓名\ \dlmu[9cm]{陈文轩} \qquad  \\ %学号
        \vspace{0.5cm}
        \kaishu\Large 专业\ \dlmu[9cm]{智能硬件与系统(电子信息工程)} \qquad \\ %姓名 
    }
        
    \date{\today} % 默认为今天的日期,可以注释掉不显示日期
}
%%------------------------document环境开始------------------------%%
\begin{document}

%%-----------------------封面--------------------%%
\cover
\thispagestyle{empty} % 首页不显示页码
%%------------------摘要-------------%%
%\newpage
%\begin{abstract}




%\end{abstract}

%\thispagestyle{empty} % 首页不显示页码

%%--------------------------目录页------------------------%%
% \newpage
% \tableofcontents
% \thispagestyle{empty} % 目录不显示页码

%%------------------------正文页从这里开始-------------------%
\newpage
\setcounter{page}{1} % 让页码从正文开始编号

%%可选择这里也放一个标题
%\begin{center}
%    \title{ \Huge \textbf{{标题}}}
%\end{center}

\section{问题1}

\textbf{Linux系统中的/proc目录与其他目录相比有什么特殊之处?}

\begin{itemize}
    \item /proc 目录是一个\textbf{虚拟文件系统}(Virtual File System) \cite{man7}  
    \item \textbf{不占用磁盘空间}:/proc 中的文件并不实际存储在磁盘上,而是由内核动态生成 \cite{kernel-doc}  
    \item \textbf{实时反映系统状态}:内容随系统运行状态实时变化
    \item \textbf{特殊权限}:大多数文件即使root用户也无法直接修改 \cite{ubuntu}  
    \item \textbf{特殊访问方式}:只能通过特定工具(如cat)读取,不能用普通编辑器修改
    \item \textbf{内核接口}:提供了用户空间与内核交互的接口 \cite{ibm}  
\end{itemize}

\section{问题2}

\textbf{在/proc目录中包含的数字目录是什么作用?选择其中一个数字目录及其5个内容(文件或子目录)为例说明作用。}

\begin{itemize}
\item /proc下的数字目录对应系统中正在运行的\textbf{进程ID(PID)} \cite{tldp}  
\item 每个数字目录包含对应进程的详细信息和控制接口
\item 以PID为1的init进程目录为例 \cite{man7}  :
\begin{itemize}
\item \textbf{/proc/1/cmdline}:显示进程的完整启动命令及参数
\item \textbf{/proc/1/status}:包含进程状态、内存使用、信号掩码等信息
\item \textbf{/proc/1/fd}:目录包含进程打开的所有文件描述符
\item \textbf{/proc/1/cwd}:符号链接指向进程的当前工作目录
\item \textbf{/proc/1/exe}:符号链接指向进程的可执行文件路径
\end{itemize}
\end{itemize}

\section{问题3}

\textbf{/proc下的文件都有什么作用或者信息?例举其中5个文件说明。}

\begin{itemize}
\item \textbf{/proc/cpuinfo}:详细记录CPU信息(型号、核心数、缓存大小、频率等)\cite{kernel-doc}  
\item \textbf{/proc/meminfo}:全面统计系统内存使用情况(总量、空闲、缓存等)\cite{ibm}  
\item \textbf{/proc/mounts}:列出当前所有挂载的文件系统及其挂载参数 \cite{ubuntu}  
\item \textbf{/proc/version}:显示当前运行的内核版本和编译信息 \cite{man7}  
\item \textbf{/proc/net/dev}:提供网络接口的详细统计信息(收发包数、错误数等)\cite{tldp}
\end{itemize}

\section{问题4}

\textbf{/proc与Linux内核有什么关系?}

\begin{itemize}
\item \textbf{内核信息窗口}:/proc是内核向用户空间暴露运行时信息的标准接口 \cite{kernel-doc}  
\item \textbf{动态数据映射}:文件内容直接映射内核数据结构和系统状态 \cite{ibm}  
\item \textbf{内核调优入口}:通过/proc/sys/目录可以动态调整内核参数 \cite{man7}  
\item \textbf{进程管理基础}:为系统监控工具(如ps、top)提供进程数据源 \cite{tldp}  
\item \textbf{调试诊断接口}:开发者可通过/proc获取内核运行时的调试信息 \cite{ubuntu}  
\end{itemize}

\reference




\end{document}